\documentclass[11pt]{letter}

\usepackage[brazil]{babel}
\selectlanguage{brazil}
\usepackage[T1]{fontenc}
\usepackage[utf8]{inputenc}
\usepackage{todonotes}
\usepackage{verbatim}

\usepackage{blindtext}
\address{Departamento de Computação - \\Universidade Federal de Ouro Preto (UFOP)}
\signature{Alunos de Pós-Graduação em Ciência da Computação}
\date{29 de julho de 2016} 

\begin{document}
\begin{letter}{Aos membros que compõem o Programa de Pós-Graduação}
\opening{Carta aos membros do Colegiado do Programa de Pós-Graduação em Ciência da Computação.}

Esta carta tem o propósito de expor ao Colegiado assuntos que visam a melhoria da excelência do Programa de Pós-Graduação em Ciência da Computação e também a harmonia entre docentes e discentes do programa. O intuito desta carta não é uma ofensa aos docentes e demais profissionais que tanto se dedicam a esta grandiosa instituição, que encontra-se entre as melhores 50 universidades da América Latina, e ao excelente Programa de Pós-Graduação em Ciência da Computação. Compreendemos que nossa instituição e Programa de Pós-Graduação só alcançou estes patamares devido à capacidade destes profissionais. Nós discentes valorizamos e agradecemos a qualidade e dedicação dos profissionais disponíveis mesmo diante da desvalorização dos mesmos e algumas vezes falta de recursos por parte do Estado.

Entretanto, apesar da excelente estrutura oferecida pelo Programa de Pós-Graduação e pela Universidade Federal de Ouro Preto, questionamos a qualidade, compromisso e comportamento dos docentes em relação as aulas ministradas. Desta maneira, gostaríamos que os seguintes pontos fossem levados a debate pelos representantes dos docentes e discentes:


\begin{itemize}
	\item A qualidade de algumas aulas ministradas está aquém da capacidade profissional dos docentes e não atendem as necessidades da maioria dos discentes que encontram-se regularmente inscritos no programa. Parte dos docentes não tem conseguido transmitir/ensinar/orientar de forma plena, abrangente e eficaz o conteúdo das disciplinas. Este fato é grave e preocupante, visto que se trata de um Programa de Pós-Graduação no qual os discentes foram cuidadosamente selecionados por critérios definidos pelo próprio Programa de Pós-Graduação. Além disso, questiona-se também a aptidão de alguns docentes as disciplinas a eles designadas. Para que sejam formados bons pesquisadores e melhorar a excelência do Programa, deve-se discutir este assunto para o bem da formação de alunos capacitados;
    
    \item A pontualidade e assiduidade dos docentes em relação as disciplinas ministradas, sendo este ato um desrespeito com os demais alunos;

	\item O comportamento dos docentes em relação aos discentes do programa de pós-graduação. Este comportamento deve ser harmonioso e saudável, não conduzidos por superioridade e falta de respeito entre as partes como vem acontecendo nas aulas atualmente. Independente do desempenho dos alunos nas disciplinas, os docentes devem considerar que o processo de ensino se passa por orientação, direcionamento e transmissão do conteúdo de acordo com a necessidade dos discentes e não por opressão e falta de respeito. Não deve existir hierarquia na organização que compõe o Programa já que ambos os lados necessitam um do outro para seus fins.

\end{itemize}

Como nós somos parte desta organização, possuímos experiência para propor alguns itens que visa aumentar a excelência deste programa. Desta forma, propomos que os pontos citados anteriormente sejam amplamente debatidos pelos representantes dos docentes e discentes para melhoria da excelência do Programa da Pós-Graduação sempre que possível. Sugerimos também que esta comissão avalie as seguintes sugestões propostas pelos alunos:

\begin{itemize}
	\item Seleção dos docentes mais aptos as disciplinas oferecidas pelo programa além da avaliação de sua didática;
	\item Possíveis formas de gravação e disponibilização das aulas ministradas pelos docentes, visto que este fato é uma tendência nas melhores universidades do mundo;
	\item Entende-se que em determinados momentos não há real necessidade de ministrar aulas, mas para estes casos os docentes devem deixar explícito de forma clara e objetiva avisos prévios além de orientações de estudos, por meio de e-mails ou anotações de aula;
	\item Avaliação obrigatória por parte dos discentes regularmente inscritos no Programa de Pós-Graduação em Ciência da Computação, incluindo estrutura, organização e docentes e exibição destes ao público para que sempre possam ser discutido melhorias dentro da organização nos semestres seguintes.
\end{itemize}

Gostaríamos mais uma vez ressaltar que não questionamos a qualidade e dedicação dos profissionais que compõem este Programa de Pós-Graduação. Entretanto, buscamos a melhoria do nosso ensino compreendendo que ele precisa ser "reinventado" e amplamente discutido com o passar dos anos.

\closing{Assim, encerramos esta carta, sua discussão e propostas.}
\end{letter}
\end{document}